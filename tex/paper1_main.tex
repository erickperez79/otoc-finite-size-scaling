\documentclass[twocolumn,preprintnumbers,amsmath,amssymb,pra]{revtex4-2}
\usepackage{graphicx,amsmath,amssymb,hyperref,xcolor,booktabs}
\begin{document}
% \preprint{}

\title{Probing Quantum Scrambling via OTOCs in Digital Circuits:\\
  Finite-Size Recurrences, System-Size Scaling,\\
  and Validation on IBM Quantum Hardware}

\author{Erick Francisco P\'erez Eugenio}
\email{erick.fpe79@gmail.com}
\affiliation{Independent Researcher,
  ORCID: \href{https://orcid.org/0009-0006-3228-4847}{0009-0006-3228-4847}}
\date{February 2026}

\begin{abstract}
We study quantum scrambling via out-of-time-ordered correlators (OTOCs)
in the Kicked Ising model across system sizes $N = 4, 8, 12, 20$,
combining exact statevector simulation with experiments on IBM Quantum
hardware (ibm\_marrakesh, 156 qubits). At $N = 4$ (Hilbert space
dimension~16), quantum recurrences completely prevent exponential OTOC
decay ($R^2 \approx 0$); remarkably, these recurrences---including a
prominent revival at depth $d = 4$---are directly observed on hardware
(Pearson $r = 0.91$ between exact and experimental patterns). As $N$
increases, recurrences are exponentially suppressed: exact simulation at
$N = 20$ reveals clean exponential decay over more than 10 orders of magnitude
($R^2 = 0.97$, $\lambda_L = 3.12 \pm 0.1$). Hardware experiments
confirm the monotonic decrease of scrambling residual $\Omega$ with $N$,
and reproduce integrable (Clifford) circuit patterns with absolute error $< 0.02$.
We compare four models---Kicked Ising, integrable, Floquet prethermal,
and disorder-averaged SYK---establishing that OTOCs reliably distinguish
dynamical regimes at all system sizes, but Lyapunov exponent extraction
requires $N \gg 4$. These results provide concrete benchmarks for
NISQ-era scrambling experiments.
\end{abstract}
\maketitle

%=================================================================
\section{Introduction}\label{sec:intro}
%=================================================================

Quantum scrambling---the delocalization of initially localized quantum
information across a system's degrees of freedom---connects quantum
chaos, black hole physics, and quantum information theory
\cite{Maldacena2016,Swingle2018,XuSwingle2024}. The OTOC provides the
standard quantitative probe: in chaotic systems with a semiclassical
limit, $C(t) \sim e^{-\lambda_L t}$, where $\lambda_L$ is the quantum
Lyapunov exponent, bounded by $\lambda_L \leq 2\pi T$ (the MSS bound)
\cite{Maldacena2016}.

The Kicked Ising model is a paradigmatic example of many-body quantum
chaos \cite{BertiniKosProsen2018,BertiniKosProsen2019}, and its
dual-unitary limit admits exact analytical results. However, Craps
et~al.\ \cite{Craps2020} showed that spin-1/2 chains present a
fundamental challenge: the OTOC saturates before exponential growth
manifests, with the exponential window opening only at higher spin.

We address the complementary question: for fixed spin $s = 1/2$, how
does the number of sites $N$ control the observability of exponential
OTOC decay? This question has acquired practical urgency with the
recent demonstration of OTOC(2) measurements on 65 qubits by Google
Quantum AI \cite{GoogleEchoes2025} and earlier experimental OTOC
measurements on superconducting hardware \cite{Braumueller2022}.
Our approach combines exact statevector simulation (up to
$N = 20$, dimension $\sim 10^6$) with experiments on IBM Quantum
hardware (ibm\_marrakesh, 156 superconducting qubits).

Our central findings are: (i)~quantum Poincar\'e recurrences, not
decoherence, are the limiting factor for OTOC-based scrambling
measurement at small $N$; (ii)~these recurrences are directly observable
on current hardware; (iii)~exponential decay emerges cleanly at $N = 20$
in exact simulation; and (iv)~hardware experiments validate the exact
simulations wherever signal exceeds the noise floor.


%=================================================================
\section{OTOC Protocol and Diagnostics}\label{sec:otoc}
%=================================================================

We use the forward-backward OTOC protocol. Define the echo state
\begin{equation}
  |\phi(d)\rangle = U_F^{-d}\,X_0\,U_F^d\,|\psi_0\rangle,
  \label{eq:echo}
\end{equation}
where $X_0$ is the butterfly (Pauli-$X$ on qubit~0) and $U_F$ is the
Floquet operator. The OTOC signal is defined as
\begin{equation}
  C(d) = \langle\phi(d)|\,P_0\,|\phi(d)\rangle,
  \label{eq:Cd}
\end{equation}
where $P_0 = |0\rangle\langle 0|_0 \otimes \mathbb{1}$ is the
projector onto $|0\rangle$ on qubit~0. In practice, $C(d)$ is the
probability of outcome $|0\rangle$ when measuring $Z_0$ after the
echo circuit. With initial state $|\psi_0\rangle = |-\rangle\otimes
|0\rangle^{\otimes(N-1)}$, the butterfly acts as $X_0|-\rangle =
-|-\rangle$, giving $C(0) = |\langle 0|-\rangle|^2 = 1/2$ exactly.
Here $d$ is the circuit depth (number of Floquet steps).

\subsection{Scrambling residual}

To characterize the overall degree of scrambling, we define the
\emph{scrambling residual}:
\begin{equation}
  \Omega \equiv \frac{\langle C(d) \rangle_d}{C(0)},
\end{equation}
where $\langle \cdot \rangle_d$ denotes the arithmetic mean over all
measured depths $d \geq 1$. The residual satisfies $\Omega \in [0, 1]$:
$\Omega = 0$ for complete scrambling and $\Omega = 1$ for no scrambling.
This provides a single-number diagnostic that is robust against
shot noise and does not require fitting assumptions.

\subsection{Finite-size recurrences}

The Poincar\'e recurrence time scales as $\tau_P \sim 2^N$. For $N = 4$
($2^N = 16$), recurrences occur within a few Floquet steps. For
$N = 20$ ($2^N \sim 10^6$), they are negligible at accessible depths.


%=================================================================
\section{Models}\label{sec:models}
%=================================================================

We study four circuit models spanning distinct dynamical regimes.

\textbf{Kicked Ising} (chaotic):
$U_F = e^{-ih\sum_i X_i}\,e^{-iJ\sum_i Z_iZ_{i+1}}$
with $J = 0.9$, $h = 0.7$, periodic boundary conditions
($Z_{N+1} \equiv Z_1$). This lies
14.6\% above the dual-unitary point $J = \pi/4$
\cite{BertiniKosProsen2019}.

\textbf{Integrable} (Clifford):
$U_F = \prod_i \text{CNOT}_{i,i+1} \cdot \prod_i H_i$.

\textbf{Floquet prethermal}:
$RX$--$RY$ rotations, $RZZ$ coupling, and $CZ$ entangling layers
with $\theta = 0.8$, $\phi = 1.2$, $J = 0.9$.

\textbf{SYK-inspired} (disorder-averaged):
Random all-to-all $ZZ$ couplings in $[0.5, 1.5]$, averaged over
50 disorder realizations (exact simulation) or 9 realizations (hardware).


%=================================================================
\section{Exact Simulation Results}\label{sec:exact}
%=================================================================

All exact simulations use statevector evolution with complexity
$O(2^N)$ per gate. For all models, $C(0) = 0.500000$ to machine
precision ($< 10^{-10}$).

\subsection{$N = 4$: Recurrence-dominated dynamics}

Table~\ref{tab:N4} shows $C(d)$ for all four models at $N = 4$.
The Kicked Ising values oscillate erratically between $10^{-4}$ and
$0.24$, with a prominent recurrence at $d = 4$ ($C = 0.243$) and
$d = 12$ ($C = 0.213$). No exponential fit is meaningful ($R^2
\approx 0$). The integrable circuit shows perfect periodicity, and
the Floquet model oscillates similarly to the Kicked Ising---at
$N = 4$, these are indistinguishable by any OTOC diagnostic.
Figure~\ref{fig:allmodels} shows the comparison between exact
simulation and IBM hardware for all four models.

\begin{figure*}[t]
\centering
\includegraphics[width=\textwidth]{figures/fig1_otoc_all_models.png}
\caption{OTOC $C(d)$ for all four models at $N = 4$. (a)~Kicked Ising:
  erratic oscillations from quantum recurrences; IBM (red squares, 5-run
  average with error bars) reproduces the pattern with Pearson $r = 0.91$.
  (b)~Integrable (Clifford): perfect periodicity reproduced by IBM with
  error $< 0.02$. (c)~Floquet: oscillations correlated with exact
  simulation ($r = 0.96$). (d)~SYK: disorder averaging over 9 hardware
  realizations (black squares) suppresses recurrences visible in
  individual seeds (gray).}
\label{fig:allmodels}
\end{figure*}

\begin{table}[h]
\caption{$C(d)$ at $N = 4$ (exact simulation). SYK: 50-seed average.}
\label{tab:N4}
\begin{ruledtabular}
\begin{tabular}{rcccc}
$d$ & KI & Integ. & Floquet & SYK \\
\hline
1  & 0.014 & 0.000 & 0.000 & 0.234 \\
2  & 0.013 & 0.500 & 0.342 & 0.121 \\
3  & 0.048 & 0.500 & 0.261 & 0.076 \\
4  & 0.243 & 0.500 & 0.251 & 0.077 \\
5  & 0.077 & 0.000 & 0.200 & 0.082 \\
6  & 0.019 & 0.500 & 0.376 & 0.066 \\
7  & 0.088 & 0.000 & 0.116 & 0.059 \\
8  & 0.000 & 0.500 & 0.040 & 0.065 \\
10 & 0.025 & 0.500 & 0.062 & 0.055 \\
12 & 0.213 & 0.500 & 0.151 & 0.061 \\
14 & 0.007 & 0.500 & 0.163 & 0.058 \\
\end{tabular}
\end{ruledtabular}
\end{table}

\subsection{Scaling from $N = 4$ to $N = 20$}

Table~\ref{tab:scaling} shows the dramatic effect of system size on
Kicked Ising OTOCs. Three features emerge:

\textbf{(i) Locality:} $C(1)$ and $C(2)$ are identical for all $N$ to
machine precision---the first two Floquet steps affect only local qubits,
and the perturbation has not reached the far boundary.

\textbf{(ii) Convergence:} For $d = 3$--$6$, values at $N \geq 8$
agree, while $N = 4$ differs by orders of magnitude, marking the onset
of finite-size effects.

\textbf{(iii) Exponential decay at $N = 20$:} $C(d)$ decreases
monotonically from $1.3 \times 10^{-2}$ to $2.3 \times 10^{-13}$---a
span of more than 10 orders of magnitude---before the first (amplitude
$\sim 10^{-12}$) recurrence at $d = 12$. A log-linear fit over
$d = 2$--$10$ yields:
\begin{equation}
  \lambda_L = 3.12 \pm 0.1, \quad R^2 = 0.973,
  \label{eq:lambda_N20}
\end{equation}
compared to a power-law fit $R^2 = 0.926$.

\begin{table}[h]
\caption{Kicked Ising $C(d)$ vs system size (exact simulation).}
\label{tab:scaling}
\begin{ruledtabular}
\begin{tabular}{rcccc}
$d$ & $N\!=\!4$ & $N\!=\!8$ & $N\!=\!12$ & $N\!=\!20$ \\
\hline
1  & 1.4e-2 & 1.4e-2 & 1.4e-2 & 1.4e-2 \\
2  & 1.3e-2 & 1.3e-2 & 1.3e-2 & 1.3e-2 \\
3  & 4.8e-2 & 3.6e-5 & 3.6e-5 & 3.6e-5 \\
4  & 2.4e-1 & 2.3e-5 & 2.3e-5 & 2.3e-5 \\
5  & 7.7e-2 & 4.1e-6 & 1.5e-6 & 1.5e-6 \\
6  & 1.9e-2 & 1.1e-3 & 8.9e-8 & 8.9e-8 \\
7  & 8.8e-2 & 1.3e-2 & 2.5e-7 & 2.1e-10 \\
8  & 1.8e-4 & 6.2e-3 & 2.3e-6 & 1.2e-11 \\
10 & 2.5e-2 & 9.9e-3 & 3.0e-4 & 2.3e-13 \\
12 & 2.1e-1 & 6.7e-5 & 6.4e-3 & 1.6e-12 \\
14 & 6.9e-3 & 7.7e-4 & 3.4e-3 & 6.4e-10 \\
\end{tabular}
\end{ruledtabular}
\end{table}

The recurrence onset depth $d^*$ scales approximately as $d^* \sim
N/v_B$, where $v_B$ is the butterfly velocity: $d^* \approx 3$ ($N=4$),
$\approx 7$ ($N=8$), $\approx 10$ ($N=12$), and $\geq 12$ ($N=20$).
Figure~\ref{fig:scaling} shows the IBM data on a logarithmic scale
alongside the exact simulation lines, and the convergence of $\Omega$
with system size.

\begin{figure*}[t]
\centering
\includegraphics[width=\textwidth]{figures/fig2_scaling_N.png}
\caption{Kicked Ising scaling with system size. (a)~$C(d)$ vs depth on
  logarithmic scale for IBM data (symbols with error bars) overlaid on
  exact simulation (lines) at $N = 4, 8, 12, 20$. The signal drops
  monotonically faster with increasing $N$. (b)~Scrambling residual
  $\Omega$ vs $N$: both exact (black) and IBM (blue) show convergence
  toward complete scrambling, with a $\sim 1/2^N$ guide (dashed).}
\label{fig:scaling}
\end{figure*}

We note that $\lambda_L = 3.12 \approx \pi$. An independent calculation
of $T_\text{eff}$ via first-order Magnus expansion of the Floquet
operator yields a \emph{negative} effective temperature ($\beta =
-0.52$), as the initial state lies in the upper portion of the
effective Hamiltonian spectrum. This rules out MSS saturation as the
explanation. The Magnus expansion is, moreover, unreliable at these
coupling strengths (second-order corrections are 156\% of the
first-order term). We record this numerical coincidence without
claiming physical significance.


%=================================================================
\section{IBM Quantum Experiments}\label{sec:ibm}
%=================================================================

\subsection{Setup}

Experiments were performed on ibm\_marrakesh (156 superconducting qubits,
Heron processor) using Qiskit~2.3.0 and the Sampler primitive.
For each model and system size, we execute 5 independent runs at 4096
shots per circuit, yielding 429 experimental data points. Job IDs are
listed in the Supplemental Material.

\subsection{Kicked Ising: IBM vs exact simulation}

Table~\ref{tab:ibm_ki4} compares IBM results with exact simulation
for the Kicked Ising model at $N = 4$.

\begin{table}[h]
\caption{Kicked Ising $N = 4$: IBM (5-run mean $\pm$ std) vs exact.}
\label{tab:ibm_ki4}
\begin{ruledtabular}
\begin{tabular}{rcccc}
$d$ & Exact & IBM mean & $\sigma$ & $\Delta$ \\
\hline
 1 & 0.014 & 0.020 & 0.003 & +0.005 \\
 2 & 0.013 & 0.043 & 0.003 & +0.030 \\
 3 & 0.048 & 0.052 & 0.004 & +0.004 \\
 4 & 0.243 & 0.191 & 0.009 & $-$0.051 \\
 5 & 0.077 & 0.070 & 0.007 & $-$0.007 \\
 6 & 0.019 & 0.040 & 0.012 & +0.021 \\
 7 & 0.088 & 0.087 & 0.011 & $-$0.001 \\
 8 & 0.000 & 0.029 & 0.003 & +0.029 \\
10 & 0.025 & 0.052 & 0.015 & +0.027 \\
12 & 0.213 & 0.098 & 0.013 & $-$0.115 \\
14 & 0.007 & 0.050 & 0.004 & +0.043 \\
\end{tabular}
\end{ruledtabular}
\end{table}

The recurrence structure is reproduced in hardware: the prominent
revival at $d = 4$ ($C_\text{IBM} = 0.191 \pm 0.009$) and the
secondary revival at $d = 7$ ($C_\text{IBM} = 0.087 \pm 0.011$)
are clearly visible. The overall pattern correlation is Pearson
$r = 0.91$, Spearman $\rho = 0.89$.

At larger $N$, the IBM results confirm the scaling trend. For $N = 12$,
the initial decay from $C(1) = 0.022$ to $C(3) = 0.002$ is resolved
before the signal enters the noise floor at $d \geq 4$. For $N = 20$,
the signal drops below shot-noise resolution by $d = 3$.

\subsection{Integrable circuit: benchmark}

The integrable (Clifford) circuit provides a stringent benchmark
because its OTOC pattern is exactly known: $C = 0$ at depths
$d = 1, 5, 7$ and $C = 0.5$ for all other measured depths (consistent
with the Clifford circuit periodicity). IBM reproduces this with
maximum deviation $0.018$ across all depths---the clearest validation
that the protocol measures the intended quantity.

\subsection{Model classification on hardware}

Table~\ref{tab:omega} shows the scrambling residual $\Omega$ for all
models.

\begin{table}[h]
\caption{Scrambling residual $\Omega = \langle C \rangle / C(0)$ from
  exact simulation and IBM hardware.}
\label{tab:omega}
\begin{ruledtabular}
\begin{tabular}{lccl}
Model & $\Omega_\text{exact}$ & $\Omega_\text{IBM}$ & Regime \\
\hline
KI $N=20$    & 0.005 & 0.004 & Full scrambling \\
KI $N=12$    & 0.006 & 0.007 & Full scrambling \\
KI $N=8$     & 0.009 & 0.016 & Full scrambling \\
KI $N=4$     & 0.136 & 0.133 & Strong scr. \\
SYK $N=4$    & 0.173 & 0.180 & Intermediate \\
Floquet $N=4$ & 0.357 & 0.268 & Intermediate \\
Integ. $N=4$ & 0.727 & 0.726 & No scrambling \\
\end{tabular}
\end{ruledtabular}
\end{table}

The classification is fully consistent between simulation and hardware.
The maximum discrepancy is $|\Delta\Omega| = 0.089$ (Floquet); for all
other models, $|\Delta\Omega| < 0.01$. Figure~\ref{fig:classification}
visualizes this comparison.

\begin{figure}[h]
\centering
\includegraphics[width=\columnwidth]{figures/fig3_classification.png}
\caption{Scrambling classification: $\Omega$ for all models from exact
  simulation (dark bars) and IBM hardware (blue bars). Background shading
  indicates regime boundaries. The ordering is fully consistent between
  simulation and experiment.}
\label{fig:classification}
\end{figure}

\subsection{Noise floor analysis}

The hardware noise floor for $C(d \geq 6)$ scales as approximately
$1.5/2^N$: 0.059 ($N=4$), 0.005 ($N=8$), 0.0005 ($N=12$), consistent
with depolarization toward the maximally mixed state. For $N = 20$,
this floor falls below shot-noise resolution.
Figure~\ref{fig:noisefloor} characterizes this scaling and decomposes
the $N = 4$ signal into physical and noise contributions.

\begin{figure*}[t]
\centering
\includegraphics[width=\textwidth]{figures/fig4_noise_floor.png}
\caption{Hardware noise floor characterization. (a)~Deep-circuit noise
  floor $\langle C(d \geq 6)\rangle$ vs system size $N$, comparing IBM
  (blue), exact simulation (black), and $1/2^N$ uniform noise (gray).
  (b)~Signal-noise decomposition for KI $N = 4$: exact $C(d)$ (dark
  bars) vs $|$IBM $-$ exact$|$ (pink bars). The dashed line marks
  $1/2^4 = 0.0625$; signal dominates noise at recurrence peaks
  ($d = 4, 7, 12$).}
\label{fig:noisefloor}
\end{figure*}


%=================================================================
\section{SYK Disorder Averaging}\label{sec:syk}
%=================================================================

The SYK model at $N = 4$ presents a qualitatively different route
to suppressing finite-size recurrences: disorder averaging over
random coupling realizations.

\subsection{Convergence with number of seeds}

We verify convergence of $\Omega$ with the number of disorder
realizations in exact simulation. With 9 seeds, $\Omega = 0.179$
is within 3.4\% of the 50-seed value ($\Omega = 0.173$), confirming
that our IBM dataset (9 completed realizations,
$\Omega_\text{IBM} = 0.180$) is sufficient for reliable classification.
Single-seed values fluctuate widely ($\Omega$ ranges from 0.04 to 0.34),
demonstrating that disorder averaging is essential.

\subsection{Functional form of the decay}

Unlike the Kicked Ising at $N = 20$, the disorder-averaged SYK at
$N = 4$ does \emph{not} exhibit clean exponential decay. An
exponential fit to the 50-seed average yields $R^2 = 0.61$, while
a power-law fit gives $R^2 = 0.91$. Individual realizations are
worse: the median single-seed exponential $R^2$ is 0.27, and only
2 of 50 seeds achieve $R^2 > 0.9$.

This is consistent with the expectation that the SYK model at finite
$N$ does not saturate the MSS bound; exponential OTOC decay requires
the large-$N$ limit. At $N = 4$, disorder averaging suppresses erratic
recurrences but does not produce the sharp exponential characteristic
of the thermodynamic limit.

Crucially, the scrambling residual $\Omega$ correctly classifies SYK
as intermediate scrambling ($\Omega = 0.173$) regardless of whether
the decay is exponential, power-law, or neither. This reinforces the
utility of $\Omega$ as a fit-free diagnostic independent of the
functional form of $C(d)$.


%=================================================================
\section{Discussion}\label{sec:discussion}
%=================================================================

\subsection{Relation to prior work}

Our findings connect to three lines of recent research.

First, Craps et~al.\ \cite{Craps2020} showed that spin-1/2 Ising
chains do not exhibit Lyapunov growth even at strongly chaotic
parameters, with the exponential window opening only at higher spin
$s$. We demonstrate the complementary mechanism: increasing $N$ at
fixed $s = 1/2$ opens the same window via exponential suppression of
Poincar\'e recurrences, rather than approach to the semiclassical
limit. Together, these results establish that OTOC-based Lyapunov
extraction requires \emph{either} large spin \emph{or} large system
size.

Second, Toga, Samlodia, and Kemper \cite{TogaSamlodiaKemper2025}
recently demonstrated MSS bound saturation in an Ising model with
site-dependent couplings derived from the AdS$_2$ metric---one of the
few known examples with only local interactions. The connection
between Ising-type models, emergent geometry, and maximal scrambling
will be explored elsewhere.

Third, and most directly relevant, Google Quantum AI
\cite{GoogleEchoes2025} demonstrated measurement of second-order OTOCs
(OTOC(2)) on 65 qubits of the Willow processor, achieving a 13{,}000$
\times$ speedup over classical simulation and claiming verifiable
quantum advantage. Their work confirms that scrambling diagnostics
are experimentally accessible at large $N$ on current hardware. Our
work addresses the complementary question: what is the \emph{minimum}
system size required for meaningful OTOC-based scrambling measurement?
While Google operated deep in the regime where scrambling is clean
($N = 65$), we systematically map the transition from
recurrence-dominated dynamics ($N = 4$) to clean exponential decay
($N = 20$), establishing the boundary that separates these regimes.

\subsection{Robustness of the exponential fit}

The exponential fit $\lambda_L = 3.12 \pm 0.1$ ($R^2 = 0.97$) at
$N = 20$ is obtained from 8 data points spanning more than 10 orders of magnitude
in exact simulation. While the distinction from a power-law fit
($R^2 = 0.93$) is suggestive, we note that with 8 points, statistical
model selection (e.g., Bayesian information criterion) cannot
conclusively distinguish exponential from stretched-exponential or
power-law with logarithmic corrections. The key observation is not
the precise functional form but the dramatic contrast with $N = 4$
($R^2 \approx 0$): increasing $N$ transforms OTOC dynamics from
incoherent oscillations to monotonic decay over many orders of
magnitude.

\subsection{What hardware can and cannot resolve}

IBM experiments reproduce OTOC dynamics when signal exceeds the noise
floor. The visibility of quantum recurrences at $N = 4$ (Pearson
$r = 0.91$) confirms the protocol measures genuine quantum dynamics,
not hardware artifacts. For $N \geq 12$, the signal enters the noise
floor within 2--3 Floquet steps; the qualitative result (rapid decay)
matches exact simulation, but the quantitative profile (exponential
rate) cannot be resolved without error mitigation.

We deliberately omit error mitigation techniques (ZNE, symmetry
verification) in this study. For $N = 4$, the raw signal-to-noise
ratio is sufficient to resolve recurrence structure. For $N \geq 12$,
the exact signal at $d \geq 5$ is $O(10^{-6})$ or smaller---no
currently available mitigation technique can recover a signal this
far below the noise floor on 4096 shots. This represents a fundamental
hardware limitation, not a methodological one.

Resolving $C(d) \sim 10^{-5}$ at $d = 5$--$10$ for $N \geq 12$ would
enable direct hardware verification of the exponential decay profile.
This could be achieved through higher shot counts ($\sim 10^8$),
improved gate fidelities, or protocols like OTOC(2)
\cite{GoogleEchoes2025} that amplify the signal via constructive
interference.

\subsection{On the metric $\Omega$}

The scrambling residual $\Omega = \langle C \rangle / C(0)$ is
deliberately simple: it requires no fitting, no model assumptions, and
is robust to shot noise. Styliaris, Anand, and Zanardi
\cite{StyliarisAnandZanardi2021} derived the equilibration value of
bipartite OTOCs from eigenstate entanglement for Haar-random unitaries;
our $\Omega$ is the experimental realization of this quantity measured
with fixed (Pauli $Z$) operators on real hardware, providing a
practical diagnostic accessible on current NISQ devices. Syzranov
et~al.\ \cite{SyzranovGorshkovGalitski2018} showed that in open
systems, OTOCs saturate to a constant set by microscopic timescales,
consistent with the finite $\Omega$ values we observe. Its main
limitation is insensitivity to
temporal structure---a system with monotonic decay and one with
oscillations of the same average amplitude yield identical $\Omega$.
For regime classification (integrable vs.\ chaotic), this suffices;
for finer diagnostics, complementary measures such as spectral
statistics or Krylov complexity \cite{Parker2019} are needed.

That $\Omega > 0$ in unitary evolution reflects a fundamental property
of quantum information: unitarity ($U^\dagger U = I$) guarantees that
information is conserved and can only be redistributed, never
destroyed. The no-hiding theorem \cite{BraunsteinPati2007} ensures that
information absent from one subsystem must reside in another, providing
a theoretical basis for the nonzero scrambling residual measured here.

We emphasize that $\Omega$ diagnoses \emph{scrambling}---the
delocalization of quantum information across a system's degrees of
freedom---rather than quantum chaos in the strict sense. Scrambling
is a broader concept: integrable systems with saddle-dominated dynamics
can exhibit exponential OTOC growth without chaos
\cite{Bhattacharjee2022}, while mixing non-chaotic regimes show
power-law OTOC decay with no well-defined Ruelle--Pollicott resonance
\cite{Znidaric2024}. In both cases, $\Omega$ remains well-defined
and informative, providing a diagnostic of informational accessibility
that is complementary to, and in some regimes more robust than,
rate-based measures \cite{GarciaMata2026}.


%=================================================================
\section{Conclusion}\label{sec:conclusion}
%=================================================================

Finite Hilbert space recurrences are the primary obstacle to OTOC-based
Lyapunov exponent extraction in small quantum systems. At $N = 4$, no
exponential decay is observable, but the recurrence pattern is
reproduced on IBM hardware with $r = 0.91$. At $N = 20$, exact
simulation reveals clean exponential decay ($R^2 = 0.97$,
$\lambda_L = 3.12 \pm 0.1$) over more than 10 orders of magnitude.

The scrambling residual $\Omega$ provides a robust, fit-free diagnostic
that correctly classifies dynamical regimes on both simulated and
hardware data. The consistency between exact simulation and 429
experimental data points on ibm\_marrakesh validates the protocol and
establishes concrete benchmarks for NISQ-era scrambling experiments.

Full details of hardware parameters, circuit structure, model
parameters, job inventory, noise floor analysis, SYK convergence
data, and code availability are provided in the Supplemental
Material~\cite{supplemental}.

\begin{acknowledgments}
We acknowledge the use of IBM Quantum services. The views expressed
are those of the author and do not reflect the official policy or
position of IBM or the IBM Quantum team.
\end{acknowledgments}

\bibliography{kaelion}
% NOTE: In Overleaf, place kaelion.bib in the same directory as this file
% or at the project root. If using subdirectories, adjust path accordingly.
% Original project path: ../../common/bibliography/kaelion

\end{document}
