% paper1_supplemental.tex — Supplemental Material for Paper 1
% Compile separately or append after \end{document} of main paper
\documentclass[aps,pra,twocolumn,superscriptaddress]{revtex4-2}
\usepackage{amsmath,amssymb}
\usepackage{graphicx}
\usepackage{booktabs}
\usepackage{longtable}
\usepackage{hyperref}
\usepackage{url}

\begin{document}

\title{Supplemental Material for:\\
Probing Quantum Scrambling via OTOCs in Digital Circuits}
\author{Erick Francisco P\'erez Eugenio}
\email{erick.fpe79@gmail.com}
\affiliation{Independent Researcher,
  ORCID: \href{https://orcid.org/0009-0006-3228-4847}{0009-0006-3228-4847}}
\date{February 2026}

\maketitle

%=================================================================
\section{IBM Quantum Hardware Details}
%=================================================================

All hardware experiments were performed on \texttt{ibm\_marrakesh},
a 156-qubit Heron processor accessed via IBM Quantum Platform.
Table~\ref{tab:hw_params} summarizes the hardware parameters at the
time of execution.

\begin{table}[h]
\caption{Hardware parameters for \texttt{ibm\_marrakesh}.}
\label{tab:hw_params}
\begin{ruledtabular}
\begin{tabular}{lc}
Parameter & Value \\
\hline
Processor & Heron \\
Total qubits & 156 \\
Median $T_1$ ($\mu$s) & $\sim$250 \\
Median $T_2$ ($\mu$s) & $\sim$150 \\
Median 2Q gate error & $\sim$0.8\% \\
Median readout error & $\sim$1.5\% \\
Qiskit version & 2.3.0 \\
Primitive & Sampler (V2) \\
Shots per circuit & 4096 \\
Runs per configuration & 5 \\
Optimization level & 3 \\
\end{tabular}
\end{ruledtabular}
\end{table}

Qubit selection was automatic via Qiskit transpilation. No manual
qubit mapping or routing optimization was applied. All circuits
used the native gate set of the Heron processor (CZ, RZ, SX, X).

%=================================================================
\section{Circuit Structure}
%=================================================================

Each OTOC measurement consists of the following steps:

\begin{enumerate}
\item \textbf{State preparation:} Apply $X$ and $H$ to qubit 0,
  preparing $|\psi_0\rangle = |{-}\rangle \otimes |0\rangle^{\otimes (N-1)}$.
\item \textbf{Forward evolution:} Apply $d$ layers of the Floquet
  circuit $U_F = U_{\text{kick}} \cdot U_{\text{Ising}}$.
\item \textbf{Perturbation:} Apply $X$ to qubit 0 (butterfly operator).
\item \textbf{Backward evolution:} Apply $d$ layers of $U_F^\dagger$.
\item \textbf{Measurement:} Measure qubit 0 in the $Z$ basis.
\end{enumerate}

The OTOC signal is extracted as $C(d) = P_0(d)$, the probability of
measuring $|0\rangle$ on qubit~0. Equivalently, $C(d) = 1 - P_1(d)$.
For a perfect echo ($d = 0$), $C(0) = |\langle 0|-\rangle|^2 = 1/2$
exactly, since the butterfly $X_0$ acts as a global phase on $|-\rangle$.

For each depth $d$, a reference circuit without the butterfly operator
is also run to verify protocol integrity.

The circuit depth (number of two-qubit gates) scales as:
\begin{itemize}
\item $d$ layers forward: $d \times (N - 1)$ CNOT equivalents for Ising,
  plus $N$ single-qubit gates for kick.
\item $d$ layers backward: identical count.
\item Total two-qubit gates: $\approx 2d(N-1)$.
\end{itemize}

For $N = 20$, $d = 7$: total $\approx 266$ two-qubit gates, well beyond
the coherence limit for faithful state evolution.

%=================================================================
\section{Model Parameters}
%=================================================================

Table~\ref{tab:model_params} lists the parameters used for each model.

\begin{table}[h]
\caption{Model parameters used in all simulations and experiments.}
\label{tab:model_params}
\begin{ruledtabular}
\begin{tabular}{lcc}
Model & Parameters & Notes \\
\hline
Kicked Ising & $J = 0.9$, $h = 0.7$ & Chaotic, PBC \\
Integrable & H + CNOT layers & Clifford, OBC \\
Floquet & $\theta = 0.8$, $\phi = 1.2$, $J = 0.9$ & Prethermal, PBC \\
SYK & $J_{ij} \sim U[0.5,1.5]$ & 50 seeds (exact), 9 (IBM) \\
\end{tabular}
\end{ruledtabular}
\end{table}

For the Kicked Ising model, the Floquet operator is:
\begin{equation}
U_F = \exp\!\left(-i h \sum_j X_j\right)
      \exp\!\left(-i J \sum_j Z_j Z_{j+1}\right)
\end{equation}
with periodic boundary conditions ($Z_{N+1} \equiv Z_1$).

For the Floquet prethermal model, each layer consists of $RX(2\theta)$
and $RY(2\phi)$ rotations on all qubits, $RZZ(2J)$ coupling on
nearest neighbors with periodic boundary conditions, and $CZ$ gates
on even-indexed pairs, with parameters $\theta = 0.8$, $\phi = 1.2$,
$J = 0.9$.

For the SYK-inspired model at $N = 4$, the Hamiltonian consists of
random all-to-all $ZZ$ couplings:
\begin{equation}
H_{\text{SYK}} = \sum_{i<j} J_{ij}\, Z_i Z_j
\end{equation}
where $J_{ij}$ are drawn independently from a uniform distribution on
$[0.5, 1.5]$. Each disorder realization uses a different set of
couplings. Time evolution is $U(t) = e^{-iH_{\text{SYK}} t}$ with
$t = 1$ per Floquet step.

%=================================================================
\section{IBM Quantum Job Inventory}
%=================================================================

A total of 40 jobs were submitted to \texttt{ibm\_marrakesh} between
February 12--14, 2026. Table~\ref{tab:jobs} provides a summary by
model and system size.

\begin{table}[h]
\caption{Job summary by model and configuration.}
\label{tab:jobs}
\begin{ruledtabular}
\begin{tabular}{lcccr}
Model & $N$ & Runs & Depths & Points \\
\hline
Kicked Ising & 4  & 5 & 11 & 55 \\
Kicked Ising & 8  & 5 & 11 & 55 \\
Kicked Ising & 12 & 5 & 11 & 55 \\
Kicked Ising & 20 & 5 & 11 & 55 \\
Integrable   & 4  & 5 & 11 & 55 \\
Floquet      & 4  & 5 & 11 & 55 \\
SYK          & 4  & 9 seeds & 11 & 99 \\
\hline
\textbf{Total} & & & & \textbf{429} \\
\end{tabular}
\end{ruledtabular}
\end{table}

Depths measured: $d \in \{1, 2, 3, 4, 5, 6, 7, 8, 10, 12, 14\}$ for
all configurations (11 data depths per run, plus $d = 0$
normalization reference).

Full job IDs are available in the data repository.

%=================================================================
\section{Exact Simulation Details}
%=================================================================

All exact simulations used statevector evolution (no sampling noise,
no approximation). The simulation computes the full $2^N$-dimensional
state vector at each Floquet step.

\begin{table}[h]
\caption{Exact simulation inventory.}
\label{tab:exact_inv}
\begin{ruledtabular}
\begin{tabular}{lccr}
Model & $N$ & Depths & Points \\
\hline
Kicked Ising & 4, 8, 12, 20 & 11 each & 44 \\
Integrable   & 4 & 11 & 11 \\
Floquet      & 4 & 11 & 11 \\
SYK (50 seeds) & 4 & 11 each & 550 \\
\hline
\textbf{Total} & & & \textbf{616} \\
\end{tabular}
\end{ruledtabular}
\end{table}

For $N = 20$, the Hilbert space dimension is $2^{20} = 1{,}048{,}576$.
Each Floquet step requires multiplication of the state vector by two
unitary matrices (Ising and kick), implemented as sparse matrix
operations. Total computation time: $\sim$30 minutes on Google Colab
(single CPU).

The Lyapunov exponent $\lambda_L = 3.12 \pm 0.1$ was extracted by
fitting $\ln C(d)$ vs $d$ for $d = 2$--$10$ at $N = 20$, excluding
$d = 0$ (normalization) and $d = 1$ (pre-scrambling, identical across
all $N$ due to locality).

%=================================================================
\section{Noise Floor Analysis}
%=================================================================

The hardware noise floor for each $N$ was estimated as the mean of
$C(d)$ for depths $d \geq 6$, where the exact signal has decayed
significantly.

\begin{table}[h]
\caption{Noise floor estimates.}
\label{tab:noise}
\begin{ruledtabular}
\begin{tabular}{cccc}
$N$ & $\langle C \rangle_{\text{IBM}}^{d \geq 6}$
    & $1/2^N$ & Ratio \\
\hline
4  & 0.059   & 0.0625 & 0.9$\times$ \\
8  & 0.005   & 0.0039 & 1.3$\times$ \\
12 & 0.0005  & 0.00024 & 1.9$\times$ \\
20 & $<$0.0001 & 0.000001 & --- \\
\end{tabular}
\end{ruledtabular}
\end{table}

The noise floor is consistent with depolarization toward the maximally
mixed state, for which $C = 0$ (each qubit equally likely to be
$|0\rangle$ or $|1\rangle$). The scaling $\sim 1.5/2^N$ reflects the
exponential suppression of any residual coherence with system size.

For $N = 20$, the shot noise floor ($1/\sqrt{4096} \approx 0.016$)
exceeds any possible signal beyond $d = 2$. This means IBM returns
$C \approx 0$ by decoherence, while exact simulation gives
$C \approx 0$ by scrambling---the same qualitative result via
different mechanisms.

%=================================================================
\section{SYK Convergence Data}
%=================================================================

Table~\ref{tab:syk_conv_full} shows the convergence of the
disorder-averaged OTOC statistics with the number of seeds.

\begin{table}[h]
\caption{SYK $N = 4$ convergence with number of disorder realizations.}
\label{tab:syk_conv_full}
\begin{ruledtabular}
\begin{tabular}{rcccc}
Seeds & $\Omega$ & $R^2_{\text{exp}}$ & $R^2_{\text{pow}}$ & $\Delta\Omega$ \\
\hline
1  & 0.342 & 0.15 & 0.22 & +97\% \\
3  & 0.253 & 0.41 & 0.55 & +46\% \\
5  & 0.203 & 0.52 & 0.72 & +17\% \\
9  & 0.179 & 0.58 & 0.85 & +3.4\% \\
15 & 0.164 & 0.60 & 0.89 & $-$5\% \\
25 & 0.170 & 0.61 & 0.90 & $-$2\% \\
50 & 0.173 & 0.61 & 0.91 & --- \\
\end{tabular}
\end{ruledtabular}
\end{table}

Individual seed statistics (50 seeds): median $R^2_{\text{exp}} = 0.27$,
mean $= 0.35$, only 2/50 with $R^2 > 0.9$. The exponential model is
not intrinsic to individual SYK realizations at $N = 4$.

%=================================================================
\section{Scrambling Residual $\Omega$ — Full Data}
%=================================================================

\begin{table}[h]
\caption{Complete $\Omega$ values for all models and system sizes.}
\label{tab:omega_full}
\begin{ruledtabular}
\begin{tabular}{lccc}
Model & $\Omega_{\text{exact}}$ & $\Omega_{\text{IBM}}$ & $|\Delta\Omega|$ \\
\hline
KI $N=4$     & 0.136 & 0.133 & 0.003 \\
KI $N=8$     & 0.009 & 0.016 & 0.007 \\
KI $N=12$    & 0.006 & 0.007 & 0.001 \\
KI $N=20$    & 0.005 & 0.004 & 0.001 \\
Integrable $N=4$ & 0.727 & 0.726 & 0.001 \\
Floquet $N=4$    & 0.357 & 0.268 & 0.089 \\
SYK $N=4$ (9s)   & 0.179 & 0.180 & 0.001 \\
SYK $N=4$ (50s)  & 0.173 & --- & --- \\
\end{tabular}
\end{ruledtabular}
\end{table}

The maximum discrepancy $|\Delta\Omega| = 0.089$ occurs for the
Floquet model, attributable to decoherence at deep circuits where
the near-integrable dynamics produce slow decay.

%=================================================================
\section{Code and Data Availability}
%=================================================================

All simulation code is written in Python using:
\begin{itemize}
\item \texttt{numpy} and \texttt{scipy} for exact statevector evolution
\item \texttt{qiskit} 2.3.0 for circuit construction and IBM execution
\item \texttt{matplotlib} for figure generation
\end{itemize}

The complete codebase, raw data (JSON format), and figure-generation
scripts are available at
\url{https://github.com/ErickPerez79/otoc-finite-size-scaling}
and archived on Zenodo (DOI to be assigned upon publication).
The repository includes:

\begin{itemize}
\item \texttt{paper1\_analisis\_ibm\_v1.py}: IBM data collection,
  exact statevector simulation, and full experimental protocol.
\item \texttt{paper1\_figuras.py}: Data recovery, comparison with
  exact simulation, statistical analysis, and figure generation.
\item \texttt{paper1\_raw\_data.json}: 616 exact simulation data points
  (all models, all $N$).
\item \texttt{paper1\_recovered\_ibm\_data.json}: 429 IBM experimental
  data points with run-level statistics and full count distributions.
\item \texttt{README.md}: Reproduction instructions for all figures
  and tables.
\end{itemize}

\bibliography{kaelion}

\end{document}
